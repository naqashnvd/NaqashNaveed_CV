% YAAC Another Awesome CV LaTeX Template
%
% This template has been downloaded from:
% https://github.com/darwiin/yaac-another-awesome-cv
%
% Author:
% Christophe Roger
%
% Template license:
% CC BY-SA 4.0 (https://creativecommons.org/licenses/by-sa/4.0/)
%Section: Work Experience at the top
\sectionTitle{Professional Experience}{\faSuitcase}
%\renewcommand{\labelitemi}{$\bullet$}
\begin{experiences}

\experience
    {June 2025}   {Master Thesis Student}{Infineon Technologies AG}{Munich, Germany}
    {December 2025} {
    \begin{itemize}
    \item Gained in-depth knowledge of control strategies, power converters and PSoC control MCU.
    \item Implemented firmware (C) for peak current mode control (PCMC) strategy for dual active bridge (DAB) topology on PSoC control MCU.
    \item Validated PCMC using PLECS and RT Box Hardware-in-the-Loop (HIL), and currently working on validation with DAB hardware.
    \item Hand's on experience with ModusToolbox and device configurator. 
    \end{itemize}
    }
    {PCMC, DAB, C, MCU, HIL, PLECS, RT-BOX, ARM, ModusToolbox}
  \emptySeparator


\experience
    {June 2024}   {Working Student: Software Engineer}{Infineon Technologies AG}{Munich, Germany}
    {May 2025} {
    \begin{itemize}
 \item Developed chaiTea Based Tests for PSoC Control MCU.
 \item Developed C-based unit tests for power conversion libraries using Unity, Ceedling, and CMock.
 \item Implemented C-based fuzz tests with AFL++ fuzzers to enhance software robustness.
 \item Integrated Docker-based test environments, optimizing execution and debugging workflows for C-based tests.
 \item Designed Simulink-based tests for power conversion libraries using Simulink Test Manager.
 \item Managed CI\textbackslash CD pipeline and improved test documentation in Confluence, TM4J, and Jira, ensuring traceability of test specifications.
    \end{itemize}
    }
    {C, Simulink, MATLAB, Git, Jira, CI\textbackslash CD, ARM, Docker, Unity, Ceedling, CMock, AFL++}
  \emptySeparator

\experience
    {October 2024}   {Internship: CVA6 Memory Encryption}{Technical University of Munich (TUM)}{Munich, Germany}
    {April 2025} {
    \begin{itemize}
\item  Developed RTL (System Verilog) for memory encryption modules for Culsans-CVA6 (RISC-V-based architecture), integrating ASCON encryption between the LLC and DRAM.
\item  Worked on cache and memory hierarchy analysis, including eviction policies and cache line.
\item  Worked on simulation environments for CVA6 multicore, enabling OpenOCD-based JTAG simulation.
\item  Created and improved Makefiles and Python scripts.
\item  CoreMark benchmarks to validate the design.
    \end{itemize}
    }
    {System Verilog, RISC-V, make, python, Modelsim, OpenOCD}
  \emptySeparator


% \experience
%     {January 2024}   {Working Student: Embedded/Electronics}{EcoG GmbH}{Munich, Germany}
%     {May 2024} {
%     \begin{itemize}
%  \item  Assisted in developing, validating, and testing embedded systems and electronics hardware.
%  \item  Contributed to circuit diagram design, and understanding of electronic components.
%  \item  Conducted soldering tasks and supported prototype testing activities.
%  \item  Analyzed field returns, conducted software and hardware testing, performed root cause analysis, and documented findings.
%     \end{itemize}
%     }
%     {Python, Altium}
%   \emptySeparator


  \experience
  {February 2021}   {System Design Engineer}{Signatics pvt ltd}{Pakistan}
  {October 2023} {
  \begin{itemize}
    \item \textbf{Signal Monitoring:} Design and implementation (Verilog, VHDL, C) of high-performance signal processing applications on a custom hardware platform (Software Defined Radio (SDR)) featuring high-speed ADCs and Xilinx Zynq Ultrascale+ (FPGA, ARM) with a two-stage superheterodyne RF front-end. Developed a real-time spectrum analyzer (30 MHz–6 GHz sweep, 500 MHz bandwidth, 1 μs time resolution) with features including persistence, peak hold, waterfall, and multi-channel channelizers. Enabled real-time data streaming to LabVIEW Host via PCIe DLLs (C/C++, Windows DLLs).
    \item \textbf{Direction Finding (DF):} Developed RTL (Verilog, VHDL) and firmware (C) for a five-channel, high-speed ADC system (500 MHz IQ sample rate) on Zynq Ultrascale+ for direction finding. Implemented phase alignment, DDC-based downsampling, and real-time data streaming to host PC via PS-Ethernet. Utilized MUSIC and Correlative Interferometry algorithms (LabVIEW, C) for direction estimation.
    \item \textbf{Digital Down Converter (DDC):} Designed and implemented a fixed DDC on LabVIEW FPGA for NI USRP TwinRX, achieving 400x downsampling and 80 dB out-of-band attenuation using three-stage poly-phase filters and DDS-based frequency shifting. Enabled phase synchronization for multi-channel TwinRX setups.
  \item \textbf{Additional Contributions:} Board bring-up and peripheral driver development (Verilog, VHDL, C); Applications for Xilinx interfaces (DMA, AXI, MIG, DDR, SERDES, PCIe, 10G Ethernet, Aurora, JESD204B); Application for Xilinx Ultrascale+ RFSoC with phase synchronize ADC and DAC; Real-time data synchronization between PL and Multicore PS; performed system-level optimization for SFDR and sensitivity; supported lab testing with oscilloscopes, spectrum analyzers, and signal generators.
  \end{itemize}
  }
  {C, C++, Xilinx Vivado, Verilog HDL, VHDL, RTL, RFSoC, Zynq Ultrascale+, SERDES, PCIe, Aurora, JESD204B, DSP, Ethernet Subsystem, Embedded C, RF, SDR, LabVIEW FPGA, Python, ARM}
  \emptySeparator
  
  \experience
  {July 2020}   {Hardware Design Engineer (Embedded)}{AKSA-SDS pvt ltd}{Pakistan}
  {February 2021} {
  \begin{itemize}
  \item \textbf{Radio Frequency Front-End (RFFE):}  Parsed Ethernet-based communication from SDR to BeagleBone MCU (Python) relayed to Xilinx Spartan 6 using SPI and UART. Designed and implemented a control unit (Verilog/VHDL, Xilinx Spartan 6) for the RF front-end of a SDR to control RF chain (local oscillators, direct digital synthesizers, ADCs, DACs, digital alternators, and amplifiers).
  \item \textbf{Standby Flight Information Display Unit (SFIDU):} Optimized memory resources and user experience for a standby altitude meter for aircraft emergency use. The system consisted of a Zynq 7000 SoC FPGA coupled with STM32, providing reliable backup flight information.
\item \textbf{Additional Contributions:} Developed a custom PetaLinux build for Xilinx SoC; implemented BPSK and MSK modulation/demodulation models in MATLAB; performed hands-on lab testing with oscilloscopes, spectrum analyzers, and signal generators.
  \end{itemize}
  }
  {C, C++, Xilinx ISE, Xilinx Vivado, Verilog HDL, VHDL, RTL, Python, Petalinux, SDR, ARM, STM32}
  \emptySeparator
  
  %  \experience
  %   {Present}   {Embedded Design Freelancer}{Fiverr}{}
  %   {April 2020} {
  %   \begin{itemize}
  %       \item Formulated Verilog, VHDL, Assembly(RISC-V, MIPS, ARM) design's
  %       \item Hands on Experience on Quartus, Xilinx ISE, Xilinx vivado, Lattice Diamond, Lattice ICEcube2 IDE's
  %       \item Hands on Experience on Modelsim, ActiveHDL, Icarus Verilog, GHDL, EDA playground, GTK wave simulator IDE's
  %   \end{itemize}
  %   }
  %   {Xilinx ISE, Verilog HDL, VHDL, RISC-V, MIPS, ARM}
  % \emptySeparator
    \experience
    {September 2019}   {Research And Development Intern}{Real-time Intelligent Secure Computing (RISC) Research Lab}{Pakistan}
    {February 2020} {
    \begin{itemize}
        \item Theory of RISC-V ISA
        \item Key concepts of Verilog HDL
        \item Hands on experience on Intel Quartus II , QSYS and ModelSim
        \item Implementation of five stage pipelined RV32I on logisim
        \item Implementation of five stage pipelined RV32I in Verilog HDL and tested it on altera DE-1 board
        \item Floating point extension in RV32I 
        \item GNU cross compiler support for RV32IF
        \item JTAG UART support for RV32IF                                                      
    \end{itemize}
    }
    {RISC V, Verilog HDL, Quartus, Modelsim, Logisim, Assembly, C, C++, Python}
  \emptySeparator
  
  %   \experience
  %   {September 2019}   {Research And Development Intern}{System Analysis and Verification (SAVe) Lab}{Pakistan}
  %   {February 2020} {
  %   \begin{itemize}
  %       \item Theory of peer to peer networks
  %       \item Theory of WiFi-Direct
  %       \item Development of Android application using Android Studio to create a WiFi direct network and establish communication between devices.
  %       \item Development of Ad-hoc network on Raspberry Pi 3 B+
  %       \item Socket programming in python
  %       \item Development of full stack desktop application in JAVA to communicate with the devices.
  %       \item Indoor localization of devices using RSSI                                              
  %   \end{itemize}
  %   }
  %   {P2P Networks, WiFi Direct, Android Studio, Raspberry Pi, JAVA, Python}
  % \emptySeparator
  
  %    \experience
  %   {July 2018}   {Research And Development Intern}{Electronic System and Design Automation Center (ESDA) Lab}{Pakistan}
  %   {August 2018} {
  %   \begin{itemize}
  %       \item Theory of Internet of Things (IoT)
  %       \item  Experience on Arduino IDE using Arduino Uno and Arduino Mega
  %       \item  Integration of water level sensors with NodeMcu
  %       \item TCP/UDP Socket programming on NodeMcu
  %       \item Data analysis and manipulation of data to compute the water-level and its flow rate using python numpy
  %       \item Integration of mobile application build using Android Studio with NodeMcu
  %       \item Integration of actuators such as motors and buzzers with NodeMcu          
  %   \end{itemize}
  %   }
  %   {IoT , Arduino, NodeMcu, Python}
  % \emptySeparator
  
\end{experiences}
